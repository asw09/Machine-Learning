\documentclass[a4paper]{article}
\usepackage{fullpage, titling, amsmath, footnote, listings}
\makesavenoteenv{tabular}
\setlength{\droptitle}{-50pt}

\title{Machine Learning Assignment 2: Decision Trees}
\author{Chris Bates, Joe Slade, Andrew West, Thomas Wood}

\begin{document}
\maketitle

\begin{itemize}
  \item The acquired decision trees, trained on the whole available dataset (6
    in total, for 6 different emotions)

    % TODO: Put trees in here/an appendix

  \item Each example needs to get only a single emotion assigned to it, between
    1 and 6. Explain how you made sure this is always the case in your decision
    tree algorithm.

  \item Have you encountered any ambiguity when attempting to classify a given
    sample, if yes, what strategies did you use to deal with this?

  \item Average cross validation classification results, that include:
    \begin{itemize}
      \item Confusion matrix
      \item Average recall and precision rates per class
      \item The $F_1$-measure derived from the recall and precision rates of the
        previous step.
    \end{itemize}
    This implies that students will have to write a small script that splits the
    given dataset into training and test sets.

  \item Pruning is an important issue in trees. Run the \emph{pruning\_example}
    function, which is provided, and briefly explain how it works. It uses the
    MATLAB built-in functions for trees. Two figures should be generated showing
    two different curves. Include those figures in your report and explain why
    each curve has this shape. What is the difference between them?
\end{itemize}

\end{document}



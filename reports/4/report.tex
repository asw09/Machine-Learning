\documentclass[a4paper]{article}
\usepackage{fullpage, titling, amsmath, footnote, listings, graphicx, subfig, lscape}
\makesavenoteenv{tabular}
\setlength{\droptitle}{-50pt}

\title{Machine Learning Assignment 5: ANOVA Test}

\author{Group 31 \\ Andrew West}

\begin{document}
\maketitle

\section{General Results (Noisy Data)}

Results of testing the various algorithms on noisy data, when trained on clean data, were as follows:\\

\begin{center}
  \begin{tabular}{c|cccccc|c}
    Algorithm & Anger & Disgust & Fear & Happiness & Sadness & Surprise & Average $F_1$\\
    \hline
    Decision Trees 	& 0.333 & 0.919 & 0.727 & 0.950 & 0.667 & 0.952 & 0.7581 \\
	ANN 			& 0.875 & 0.971 & 0.914 & 1.000	& 0.889 & 1.000 & 0.9416 \\
	CBR System 		& 0.424 & 0.000 & 0.882 & 1.000	& 0.889 & 0.952 & 0.6913 \\
  \end{tabular}\\
  \end{center}
  
From these results, it is possible to say that Neural Networks appear to be the best learning algorithm to use. This can
been seen by comparison of the average $F_1$ measures for each algorithm. As these results were produced by training on
the entire clean data set and tested on the entire noisy set, the values for each measure for the ANN system and the
Case-Based Reasoning system are deterministic (i.e. they will always produce the same average $F_1$ values), so if this
experiment were run an arbitrary number of times the means of the distributions of average $F_1$ would be 0.9416 and
0.6913 respectively - the ANN system is clearly better suited to this particular task than the CBR system. Comparison
with the Decision Tree algorithm is slightly more difficult as it produces slighlty different results each time it is run.
However, by repeating this experiment 50 times and performing a t-test on the resulting vectors of average $F_1$ values,
it was discovered that the two distributions had different means (at the 95\% significance level), with the ANN mean being
higher. Although we cannot say that the ANN learning algorithm is the best learning algorithm in general, if it is assumed
that the clean and noisy datasets provided were representative of real data then we can at least say that the ANN algorithm
is best suited to the task of classifying noisy data when trained on clean data.\\

\newpage
\section{ANOVA Test Results (Clean Data)}

The results of ANOVA, multiple comparison and t-tests can be seen below:

\begin{center}
  \begin{tabular}{|c|c|c|c|}
  \hline
    {\bf Comparison} & {\bf ANOVA p-value} & {\bf Multiple Comparison [95\% c.i.]} & {\bf T-Test (H value)} \\
    \hline
    \multicolumn{4}{|c|}{Anger} \\ \hline
	DT vs. NN 	& 			& similar [-0.5518,0.1058] 	& 0 \\
	NN vs. CBR 	& 0.0526 	& similar [-0.4399,0.2177]	& 0 \\
	DT vs. CBR 	& 			& different [-0.6629,-0.0053] & 1 \\
	\hline
    \multicolumn{4}{|c|}{Disgust} \\ \hline
	DT vs. NN 	& 			& similar [-0.0585,0.0699]	& 0 \\
	NN vs. CBR 	& 0.9682 	& similar [-0.0642,0.0642]	& 0 \\
	DT vs. CBR 	& 			& similar [-0.0585,0.0699]	& 0 \\
	\hline
	\multicolumn{4}{|c|}{Fear} \\ \hline
	DT vs. NN 	& 			& similar [-0.5899,0.3219]	& 0 \\
	NN vs. CBR 	& 0.5648 	& similar [-0.5159,0.3959]  & 0 \\
	DT vs. CBR 	& 			& similar [-0.6499,0.2619]	& 0 \\
	\hline
	\multicolumn{4}{|c|}{Happiness} \\ \hline
	DT vs. NN 	& 			& similar [-0.1793,0.0057]	& 0 \\
	NN vs. CBR 	& 0.0408 	& similar [-0.0925,0.0925]	& 0 \\
	DT vs. CBR 	& 			& similar [-0.1793,0.0057]	& 0 \\
	\hline
	\multicolumn{4}{|c|}{Sadness} \\ \hline
	DT vs. NN 	& 			& similar [-0.4147,0.1483]	& 0 \\
	NN vs. CBR 	& 0.1014 	& similar [-0.4017,0.1613]	& 1 \\
	DT vs. CBR 	& 			& similar [-0.5349,0.0281]	& 0 \\
	\hline
	\multicolumn{4}{|c|}{Surprise} \\ \hline
	DT vs. NN 	& 			& similar [-0.3220,0.0820]	& 0 \\
	NN vs. CBR 	& 0.2531 	& similar [-0.2020,0.2020]	& 0 \\
	DT vs. CBR 	& 			& similar [-0.3220,0.0820]	& 0 \\
	\hline
  \end{tabular}\\
 \end{center}
 (DT = Decision Trees, NN = Neural Networks, CBR = Case-Based Reasoning)\\
 
 Little of interest can be gained from these results, as all algorithms performed well on clean data. A difference was found
 between classifications of `Anger' made by the Decision Tree algorithm and the Case-Based system by the multiple comparison
 test, explaining the low ANOVA value for that emotion, and the t-test between those two algorithms confirms this difference.
 In most other cases, both the ANOVA/Multiple Comparison tests and the t-tests were in agreement and no differences were found
 between the algorithms classification scores. Despite having a very low ANOVA score, no differences could be found by either
 of the follow-up tests between the average $F_1$ values for `Happiness'. The only case in which the tests produced different
 results was the classification of `Sadness' by Neural Networks and the Case-Based system, although this could be a type 1 error
 introduced by the number of t-tests performed - as tests were performed with an alpha value of 0.05, approximately one in every
 20 paired t-tests would incorrectly identify a difference, so one error could likely be expected when 18 tests were performed.\\
  
\newpage
\section{Appendix}

$F_1$ results per fold from 10-fold cross-validation on clean data:\\

{\bf Decision Trees}\\

\begin{center}
  \begin{tabular}{|c|cccccccccc|}
  \hline
    Emotion & 1 & 2 & 3 & 4 & 5 & 6 & 7 & 8 & 9 & 10 \\
    \hline
    Anger 	& 0.333 & 0.8 & 0.667 & 0 & 0.667 & 1.0 & 1.0 & 1.0 & 0.0 & 1.0 \\
	Disgust & 1.0 & 1.0 & 1.0 & 1.0 & 1.0 & 1.0 & 0.857 & 1.0 & 1.0 & 1.0 \\
	Fear 	& 0.667 & 1.0 & 0.0 & 0.0 & 1.0 & 1.0 & 1.0 & 0.0 & 1.0 & 1.0 \\
	Happiness 	& 0.667 & 1.0 & 0.667 & 0.8 & 1.0 & 1.0 & 1.0 & 1.0 & 1.0 & 1.0 \\
	Sadness 	& 0.0 & 0.8 & 1.0 & 0.0 & 1.0 & 1.0 & 1.0 & 1.0 & 0.667 & 1.0 \\
	Surprise 	& 0.0 & 1.0 & 0.8 & 1.0 & 1.0 & 1.0 & 1.0 & 1.0 & 1.0 & 1.0 \\
	\hline
  \end{tabular}\\
  \end{center}
  
 {\bf Neural Network}\\

\begin{center}
  \begin{tabular}{|c|cccccccccc|}
  \hline
    Emotion & 1 & 2 & 3 & 4 & 5 & 6 & 7 & 8 & 9 & 10 \\
    \hline
    Anger 	& 1.0 & 1.0 & 1.0 & 1.0 & 1.0 & 1.0 & 0.8 & 0.889 & 0.0 & 1.0 \\
	Disgust & 1.0 & 1.0 & 1.0 & 1.0 & 0.8 & 1.0 & 1.0 & 1.0 & 1.0 & 1.0 \\
	Fear 	& 1.0 & 1.0 & 1.0 & 1.0 & 0.0 & 0.0 & 1.0 & 1.0 & 1.0 & 1.0 \\
	Happiness 	& 1.0 & 1.0 & 1.0 & 1.0 & 1.0 & 1.0 & 1.0 & 1.0 & 1.0 & 1.0 \\
	Sadness 	& 1.0 & 1.0 & 1.0 & 1.0 & 0.8 & 0.667 & 0.667 & 0.667 & 1.0 & 1.0 \\
	Surprise 	& 1.0 & 1.0 & 1.0 & 1.0 & 1.0 & 1.0 & 1.0 & 1.0 & 1.0 & 1.0 \\
	\hline
  \end{tabular}\\
  \end{center}
  
  {\bf Case-Based Reasoning}\\

\begin{center}
  \begin{tabular}{|c|cccccccccc|}
  \hline
    Emotion & 1 & 2 & 3 & 4 & 5 & 6 & 7 & 8 & 9 & 10 \\
    \hline
    Anger 	& 1.0 & 1.0 & 1.0 & 1.0 & 0.8 & 1.0 & 1.0 & 1.0 & 1.0 & 1.0 \\
	Disgust & 0.8 & 1.0 & 1.0 & 1.0 & 1.0 & 1.0 & 1.0 & 1.0 & 1.0 & 1.0 \\
	Fear 	& 0.667 & 1.0 & 1.0 & 1.0 & 0.0 & 1.0 & 1.0 & 1.0 & 1.0 & 1.0 \\
	Happiness 	& 1.0 & 1.0 & 1.0 & 1.0 & 1.0 & 1.0 & 1.0 & 1.0 & 1.0 & 1.0 \\
	Sadness 	& 1.0 & 1.0 & 1.0 & 1.0 & 1.0 & 1.0 & 1.0 & 1.0 & 1.0 & 1.0 \\
	Surprise 	& 1.0 & 1.0 & 1.0 & 1.0 & 1.0 & 1.0 & 1.0 & 1.0 & 1.0 & 1.0 \\
	\hline
  \end{tabular}\\
  \end{center}
 
\end{document}
\documentclass[a4paper]{article}
\usepackage{fullpage, titling, amsmath, footnote, listings, graphicx, subfig}
\makesavenoteenv{tabular}
\setlength{\droptitle}{-50pt}

\title{Machine Learning Assignment 3: Artificial Neural Networks}
\author{Group 31 \\ Chris Bates, Joe Slade, Andrew West, Thomas Wood}

\begin{document}
\maketitle
\section{Network Parameters}

%  What criteria have you used to choose optimal topology/parameters of
%  the networks? Compare the optimal parameters of both types of the networks
%  (the single six-output network and six single-outputs networks). Explain what
%  strategy you employed to ensure good generalisation ability of the networks
%  and overcome the problem of overfitting, if encountered (support this by
%  experimental results).

We brute forced two of the paramters using Condor, and found they had minimal effect on the output
of the neural networks. 

Somebody played around with the other parameters, and
%BULLSHIT HERE
we graphed the results of varying paramter X and saw what was best.

In order to overcome overfitting, we sacrificed two rams to the dark god of MATLAB.


%  perform 10-fold cross-validation for both types of networks. Cross-validation
%  should be performed in the same way as in Assignment 2 (with a script that
%  splits the given dataset into training and test sets). 10-fold cross-validation
%  should be performed using the optimal topology and best parameters obtained
%  in 2(a), i.e., for six-output and single-output NNs. Note that in the case of 6
%  networks, each example must be classified as one of the 6 emotions. Plot the
%  performance (only F1 measure) per fold of each network type in the same
%  figure.




%  Report also the average results of the 10-fold cross-validation (similarly to the
%  trees) for both types of networks (single six-output NN and six single-outputs
%  NNs):
%     confusion matrices,
%     recall and precision rates per class,
%       (Hint: you can derive them directly from the previously computed  confusion matrix)
%     the F1-measure derived from the recall and precision rates of the previous step.



%  Is there any difference in the classification performance of the two different
%  classification approaches. Discuss the advantages / disadvantages of using 6
%  single-output NNs vs. 1 six-output NNs.

Single output NNs well train better to one emotion, but a six-output NN will probably be 
less likely to get overtrained.

\end{document}
\documentclass[a4paper]{article}
\usepackage{fullpage, titling, amsmath, footnote, listings, graphicx, subfig}
\makesavenoteenv{tabular}
\setlength{\droptitle}{-50pt}

\title{Machine Learning Assignment 3: Artificial Neural Networks}
\author{Group 31 \\ Chris Bates, Joe Slade, Andrew West, Thomas Wood}

\begin{document}
\maketitle
\section{Network Parameters}

%  What criteria have you used to choose optimal topology/parameters of
%  the networks? Compare the optimal parameters of both types of the networks
%  (the single six-output network and six single-outputs networks). Explain what
%  strategy you employed to ensure good generalisation ability of the networks
%  and overcome the problem of overfitting, if encountered (support this by
%  experimental results).

The first requirement of the task was to decide what constituted an 'optimal' neural network. As the networks would eventually
be compared based on their performances what subjected to 10-fold cross-validation, it was decided that we should attempt to 
maximise the average $F_1$ value from between the six emotions. Thus, the 'optimal' networks would be those that would cope
best when tested on unseen data, in a similar way to the way success of the decision trees of the previous task were measured.

Following feedback from the previous task, the function used to subivide the data set for 10-fold cross-validation was improved
to group data into test and training sets randomly in order to make our results less dependent on the arrangement of the underlying data.
Thus, a secondary concern when searching for the optimal solution was that the networks be trainable in a reasonable amount of time.
This ensured that it was possible to collect large amounts of test data for each variation of each network parameter in the
limited time available, which could then be used to counteract the effects of randomness on our results and ensure our findings
were statistically significant and supported by our data.

The actual optimisation of the parameters was effectively done by brute force. By using Condor to perform large numbers of computations
beyod the scope of what would have been achievable using individual machines, it was possible to test large numbers of different values
for each parameter. These experiments were also repeated many times to ensure their accuracy. In each experiment, one parameter was varied
while the others remained fixed. This allowed us to see that impact each had on the performance of the resulting neural networks. In ideal
circumstances, one extremely large experiment in which every permutation of possible parameter combinations would have instead been run,
but because of time constraints on the task it was decided to take a more simplistic approach to optimising the network parameter.

The first parameters to be tested were the number of hidden layers and number of neurons in each layer. Neural networks were trained
with a single hidden layer containing from 1 to 50 neurons, and then with two hidden layers each of which contained up to 50 neurons.
A small number of tests with three hidden layers were attempted, but initial results showed no improvement over those with two layers. The
results of these can be seen in figures 1 & 2. Each plots the number of neurons in each hidden layer against the average $F_1$ achieved by
networks using that topology. Unfortunately, neither showed any particular trends or patterns which could be used to clearly identify the
optimum network topography. By using the graphs and the data returned by Condor we determined the optimum number of hidden layers to be
two for both types of network. The optimum number of neurons in the single six-output tree was found to be 34 in the first layer and 15 in the
second, while the optimum values for the single output trees were 48 in the first layer and 14 in the second. Beyond some simple observations,
little can be said about the effect of topography on these networks as the generally performed well regardless of the number of neurons per layer,
with very poor performances only observed from both networks when four or fewer neurons were present in their first layers. The six-output tree
does appear to be more succeptable to changes in network topography however, as the range of $F_1$ values recorded by the experiment was far larger
than that of the single-output trees, going as low as 0.2 in some cases, while the lowest $F_1$ achived by the single-output trees was around 0.6.
Both achieved fairly high maximum values of 0.9336 and 0.9312 respectively.

The next parameters tested were the various transfer functions used between layers of the networks. Although the task specification identifies
three specifically to test for - perceptron (hardlim), linear unit (purelin) and sigmoid (tansig) - it was decided that Condor could be used
to test all 14 of the available transfer functions in the Matlab Neural Netorks Toolbox. The average $F_1$ values obtained after performing
10-fold cross-validation on the generated networks with differing transfer functons were again plotted, the results of which can be seen in figure 3.
 

%  perform 10-fold cross-validation for both types of networks. Cross-validation
%  should be performed in the same way as in Assignment 2 (with a script that
%  splits the given dataset into training and test sets). 10-fold cross-validation
%  should be performed using the optimal topology and best parameters obtained
%  in 2(a), i.e., for six-output and single-output NNs. Note that in the case of 6
%  networks, each example must be classified as one of the 6 emotions. Plot the
%  performance (only F1 measure) per fold of each network type in the same
%  figure.




%  Report also the average results of the 10-fold cross-validation (similarly to the
%  trees) for both types of networks (single six-output NN and six single-outputs
%  NNs):
%     confusion matrices,
%     recall and precision rates per class,
%       (Hint: you can derive them directly from the previously computed  confusion matrix)
%     the F1-measure derived from the recall and precision rates of the previous step.



%  Is there any difference in the classification performance of the two different
%  classification approaches. Discuss the advantages / disadvantages of using 6
%  single-output NNs vs. 1 six-output NNs.

Single output NNs well train better to one emotion, but a six-output NN will probably be 
less likely to get overtrained.

\end{document}